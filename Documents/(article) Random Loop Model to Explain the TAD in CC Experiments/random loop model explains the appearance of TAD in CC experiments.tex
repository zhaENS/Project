\documentclass[12pt]{paper}
\usepackage{amsmath}
\usepackage{amssymb}
\usepackage{graphicx}
%\usepackage{hyperref}
\usepackage{color}
\usepackage{float}
\begin{document}
\title{Random Loops Can Explain the Appearance of Topologically Associating Domain in Chromosome Capture Experiments}
\maketitle

\section{Introduction}

\section{Experimental data and Methods}\label{Methods}

We use the experimental 5C data generated by Nora et al.\cite{Nora2012} for the chromosome contact frequencies of the X chromosome in a 4.5 Gb region encompassing the X inactivation center. In our work we focus on a subset of the data, including a 94,082 bp region, termed TAD D and E (see \cite{Nora2012}). 

The contact map generated by the 5C experiments describes the contact between genomic segment of variable sizes. We then follow data coarse-graining procedure performed by Giorgetti et al.\cite{Giorgetti2014} to map segment encounter data to that of beads. A bead size of 3000 bp was chosen according to the mean size of restriction segments resulted by HindIII enzyme digestion, used in the process of the 5C experiments (see \cite{Giorgetti2014} Supplementary Material). This choice of bead size resulted in coarse-grained polymer of 307 beads. 

The coarse-grained pair-wise bead encounter frequency includes 14,509 data points and was used to calculate the bead encounter probability. We use this data to construct the bead encounter frequency for each bead as a function of bead distance. Encounter frequencies for equidistant beads were averaged. We then divide the bead encounter frequency by the total number of encounters for that bead to obtain the encounter probability. We use the bead encounter probability to display the bead encounter matrix. This is a 307 by 307 matrix containing the pair-wise encounter probabilities in each of its entries.


In our simulations we use a Rouse polymer to examine different polymer conformation that give rise to the appearance of a Topologically Associating Domain (TAD). The Rouse model describes a linear polymer as a collection of massless beads connected by harmonic springs. 




The encounter probability was fitted with a function of the form  $\alpha d^{-\beta}$, with $\alpha = \frac{1}{\sum_{j=1}^k j^{-\beta}}$, $d$ is the distance in bead units, and $\beta$ is a parameter to be determined.



\section{Results}\label{Results}


\bibliographystyle{plain}
\bibliography{randomLoops} % the bibliography.bib
\end{document}