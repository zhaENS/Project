\documentclass[12pt]{article}
\usepackage{graphicx}
\usepackage{amsmath}
\usepackage{latexsym}
\usepackage{amstext}
\usepackage{array}
\usepackage{multirow}
\usepackage{stackrel}
\usepackage{caption}
\usepackage{subcaption}

%%%%%%%%%%%%%%%%%%%%%%%%%%%%%%%%%%%%%%%%%%%%%%%%%%%%%%%%%%%%%%%%%%%%%%%%%%%=
\newtheorem{proposition}{Proposition}
\newtheorem{lemma}{Lemma}
\newtheorem{theorem}{Theorem}
\newtheorem{corollary}{Corollary}
\newcommand{\Proof}{\underline{\bf Proof:}}
\newcommand{\Endproof}{~$\Box$~}
\newcommand{\mb}[1]{ \mbox{\boldmath$#1$} }
\newcommand{\ds}{\displaystyle}
\newcommand{\beq}{\begin{eqnarray}}
\newcommand{\eeq}{\end{eqnarray}}
\newcommand{\beqq}{\begin{eqnarray*}}
\newcommand{\eeqq}{\end{eqnarray*}}
\newcommand{\p}{\partial}
\newcommand{\g}{\gamma}
\newcommand{\eps}{\varepsilon}
\newcommand{\x}{\mbox{\boldmath$x$}}
\newcommand{\n}{\mbox{\boldmath$n$}}
\newcommand{\J}{\mbox{\boldmath$J$}}
%\newcommand{\v}{\mbox{\boldmath$v$}}
\newcommand{\y}{\mbox{\boldmath$y$}}
\newcommand{\z}{\mbox{\boldmath$z$}}
\font\bb=msbm10 at 12pt
\def\rR{\hbox{\bb R}}
\def\rN{\hbox{\bb N}}
\def\rQ{\hbox{\bb Q}}
\def\rZ{\hbox{\bb Z}}
\pagestyle{plain}
%%%%%%%%%%%%%%%%%%%%%%%%%%%%%%%%%%%%%%%%%%%%%%%%%%%%%%%%%%%%%%%%%%%%%%%%%%%%
\begin{document}

\title{Chromatin reconstruction and dynamics using random Loops accounting for Chromosome Capture data}
\author{O. Shukron,  L. Giogetti, E. Heard and D. Holcman$^{1}$ \footnote{ $^1$Ecole Normale Sup\'erieure, IBENS, 46 rue d'Ulm 75005 Paris, France.}}
\date{\today}
\maketitle
%%%%%%%%%%%%%%%%%%%%%%%%%%%%%%%%%%%%%%%%%%%%%%%%%%%%%%%%%%%%%%%%%%%%%%5
\begin{abstract}\label{abstract}
\end{abstract}
{\bf \noindent Keywords: Modeling, Inverse problem, First passage times,}\\

\section{Introduction}\label{section_introduction}
%> The relationship between chromosomal spatio-temporal organization and chromosomal activity is incompletely understood
[Unfinished]\\
The spatio-temporal organization of the chromatin plays an essential role in the regulation of sub-cellular activity such as gene expression\cite{cremer2001chromosome}.

%> Dynamic looping events between enhancer and promoters (like) are frequent in the nucleus and are related to the chromosomal dynamic organization

%> Chromosome Capture method creates loci contact maps, which are used to record chromosomal static looping events

%> Static looping events, coupled with dnamic polymer models, can shed light on the chromosoamal spatio-temporal organization

%> previous work on the subject
3d structure of the Igh locus was suggested by \cite{jhunjhunwala20083d} after tagging several position along the loci.

%> our work

\section{Experimental data and Methods}\label{section_experimentalDataAndMethods}

\subsection{The experimental data}\label{subsection_theExperimentalData}
%> The experimental data from 5c experiments includes a subset of 2 TADs out of the Nora et al data,we use the average data
We used the experimental 5C data generated by Nora et al.\cite{Nora2012} for the chromosome contact frequencies of the X chromosome in a 4.5 Gb region encompassing the X inactivation center. In our work we focused on a subset of the data, including a 94,082 bp region, termed TAD D and E (see \cite{Nora2012}).
Two replicates of the experiments were conducted. In our analysis we use the average of these two experimental replicates.

%> coarse graining of the data, following Giorgetti et al. allows for uniform bead spacing representation
The encounter histograms generated by the 5C experiments describes the contact between genomic loci of variable sizes over millions of nuclei. To map these contact into uniform sized segments, we followed data coarse-graining as described in \cite{Giorgetti2014}, to map segment encounter frequencies to that of evenly spaced, equal size beads. A bead size of 3000 bp was chosen according to the mean size of restriction segments resulted by HindIII enzyme digestion, used in the process of the 5C experiments \cite{Nora2012} \cite{Giorgetti2014} (Supplementary Materials). This choice of bead size resulted in coarse-grained polymer of 307 beads.

%> we use the coarse-grained representation to calculate bead encounter probability
The coarse-grained pair-wise bead encounter frequencies includes 14,509 data points and was used to calculate the bead encounter probability as a function of the distance in beads units. For each bead, equidistant encounter frequencies were averaged, and the resulting encounter frequency signal was divided by the total number of encounters, to get the encounter probability as a function of bead distance.

%> fitting the experimental encounter probability with an analytical function
For each bead $n=1..307$ we fit the experimental encounter probability signal with a function of the form 
\begin{equation}\label{equation_encounterProbabilityModel}
p_n(d)=\alpha_n d^{-\beta_n}
\end{equation}
with $p_n$ the encounter probability of bead $n$, $\alpha_n = \frac{1}{\sum_{j=1}^k j^{-\beta_n}}$, $d$ is the distance in bead units, and $\beta_n$ is a parameter to be determined by the fitting procedure.


\subsection{The polymer model}\label{subsection_thePolymerModel}
%> Rouse polymer is used to represent the chromosome
To represent the chromosome polymer and to explore the different architectures that can explain the appearance of the TADs, we chose to use the Rouse chain. The Rouse chain describes the dynamics of a linear polymer as a collection of massless beads connected by harmonic springs and driven by the thermal forces of diffusion. The corresponding system of stochastic differential equations describing the time progression of a chain of $N$ beads is given, in the 3-dimensional case, by
\begin{equation}
\frac{dR}{dt}=-\frac{3D}{b^2}KR +\sqrt{2D}\frac{dW}{dt}
\end{equation}
where, $R(t)=[R_1(t),R_2(t),..,R_N(t)]^T$ describes the 3D coordinates of $N$ beads at time $t$, $D$ is the diffusion constant, $b$ is the standard-deviation of the distance between adjacent beads of the chain, $W$ is an independent $N\times3$ Brownian motion with mean 0 and variance 1 in each component, and $K$ is the Kirchhoff bead connectivity matrix, which reflects different chain connectivities.

%> polymer with random loops were tested to give rise to the TADs
We have constructed our polymer model to have $L$ loops of random sizes. To form each loops, we have randomly chosen 2 non-neighboring beads and altered the connectivity in the Kirchhoff matrix, with the condition that no bead can participate in the formation of more than one loop.

\subsection{Simulations}\label{subsection_simulations}
%> for each number of loops of variable size,
Throughout simulations, for each fixed number of loops, the chosen beads to connect varied randomly. Such a choice was made to refer to the heterogeneity if the spatial organization inside TAD between cells, even in the same cell phase \cite{Nora2012}.

%> simulation is done until relaxation time
Simulations were always carried out until the chain's relaxation time, in which point any two beads were determined to have encountered if their distance at the end of the simulation satisfied $|R_j-R_k|<\epsilon<b,\quad  (j\ne k)$.
The chain's relaxation time is given by the slowest mode of the linear chain
\begin{equation*}
\tau =\frac{b^2}{12D\sin(\frac{\pi}{2N})}
\end{equation*}
for which the number of simulation steps performed is $\frac{\tau}{\Delta t} $. The time step, $\Delta t$ was set so to prevent simulation 'blow-ups' by demanding that the quotient of the norms of beads position at two subsequent time steps would be smaller that unity, which resulted in $\Delta t < \frac{b^2}{12D}$.

%> fitting the simulation data and comparing to the theory to experimental data
For each tested polymer connectivity we constructed the bead encounter frequencies histogram and derived the bead encounter probability from it. The bead encounter probability was then fitted similarly to the fitting in eq. \ref{equation_encounterProbabilityModel}.

%> interpertation of the mean beta value
For a linear Rouse chain with nearest neighbor interactions, the expected value of $\beta$ is $1.5$ \cite{doi1986theory}. We interpret $\beta<1.5$ as long range interaction resulting from non nearest-neighbor bead interactions.
Because the addition of non-neighboring connections to the linear chain can only increase the long range encounter probability, we have focused on interpreting the fitted values in the range $\beta<1.5$.

%> transfroming to normal coordinates

%> previous results of the MFPT (Amitai 2012)

%> the prcess of obtaining the MFET from the normal coordiantes linear chain case
%1. transfrom from spatial coordinates to normal coordinates
%2. solve the fokker planck equation (find suitable boundary conditions)
%3. express the solution as an infinite series
%4. integrate to get the CDF and calculate the survival function
%5. integrate the survival function to get the MFET




\section{Results}\label{section_results}
%> for the case of TAD D+E, beta value is below 1.5 which indicates long range interactions
\subsection{Analysis of the experimental data}\label{subsection_analysisOfTheExperimentalData}

%> calculation of the beta values by bead and the mean values will allow us to infer on structure.
To evaluate the mean encounter probability in the experimental data, we have calculated the $\beta$ value for the 3 cases of TAD D, TAD E and TAD D+E for each of the beads in those genomic region. This provides us with a basis for comparison of the results of simulations with the experimental data and to the inference on the spatial organization of the chromosome.

%> beta values by beads for the case of TAD D+E shows a pattern correlated with the TADs location and is inline with the expected beta
The calculation of $\beta$ for each bead in the case of TAD D+E resulted in a pattern which was correlated with the significant long range interactions (Figure \ref{figure_encounterProbabilityTADDAndE+fittedBeta} lower panel), represented by the peaks the encounter probability graph (Figure \ref{figure_encounterProbabilityTADDAndE+fittedBeta}, upper panel). Indeed, the mean $\beta$ value was 0.729, which is well below the expected value for the linear Rouse chain (Figure \ref{figure_encounterProbabilityTADDAndE+fittedBeta} upper panel).

%>  Long range interactions are contributed by TAD E and inter-TAD interactions.
We then turned to examine whether long range interactions stem from inter or intra-TAD polymer looping. As can be seen in Figure \ref{figure_encounterProbabilityTADD+fittedBeta} upper panel, TAD D has almost no significant long range interactions, although the mean fitted $\beta$ value was 0.71, which indicates either packed organization of TAD D or heterogeneity of the location of loops within the cell population examined in the HiC experiments.
Intra-TAD long range interaction within TAD E contribute about half of the significant long range encounter peaks in the encounter probability graph (Figure \ref{figure_encounterProbabilityTADDAndE+fittedBeta} upper panel), whereas the other half stem from inter-TAD long range interactions.

%> we turn to search for the polymer architecture which gives rise to the experimental observations
Given the calculation of the $\beta$ values from the experimental data we now turn to explore which polymer architecture give rise to the observations.

\subsection{Random fixed loops simulations}\label{subsection_randomFixedLoopsSimulations}
%> Placing loops corresponding to the peaks of the encounter data does not rescreate the observed values extracted from the experimental data
To examine if fixed loops in the polymer can recreate the TADs, we have placed connection between beads corresponding to the peaks of the encounter probability and simulated our model to relaxation time.
In a 307 beads polymer, these fixed loops were insufficient to recreate a TAD-like structure
%\textcolor{green}{[missing the mean value of $\beta$ in TAD D+E]}. 
Only localized nearest neighbors interactions emerged by this model(Figure \ref{figure_encounterProbabilityPeaksOfTheEncounterData}) which cannot account for the observed long range interaction map.

%> strong peaks at the boundaries of TADs might indicate the presence of stable loops
Although these fixed loops are insufficient by themselves to create the encounter maps expected, we noticed that on the boundaries of TADs there is a tendency to find high peaks. We have postulated that these peaks, which connects the two boundaries of a TAD, are of significance to the spatial organization and the functionality of regulatory elements within the TAD.  We therefore examined the encounter probability of a polymer having a large fixed loop between two predefined ends.

% random positions of loops reflects the heterogeneity in internal organization within TAD of different cells
Furthermore, to reflect the heterogeneity in the spatial organization of the chromatin of cells in the HiC experiment \cite{dekker2013exploring} \cite{Nora2012}, we have added loops between randomly chosen beads on the linear chain between the two boundaries we have determined for the big loop (see Figure \ref{figure_encounterProfileOneTADWithTails}).

%> experiments with one TAD
Increasing the number of internal random loops from 1 to 10, we see an encounter pattern which resembles that of a TAD (Figure \ref{figure_encounterProfileOneTADWithTails}).

%> experiment with two TADs
Next, we added a second, adjacent region, to form a loop, and sequentially added 1 to 10 internal random fixed loops in each. (Figure \ref{figure_encounterProfileTwoTADs} ) %\textcolor{green}{[simulation have to be redone]}


%> show that the data contains peaks that might correspond to hard coded big loops
%> analyze the region containing the big loop in term of eigenvalues
%> analyze the normal coordinate system to find the encounter probability
%> show that the encounter probability beta value drops quadratically with the number of internal dynamic random loops
%> show the connection between the average beta and the average number of loops
%> argue that in the resultion of data that we have, we can only say something concrete about the large loop.
%> try to find the connection between gene expression and the number of loops the model predicted
%> try to estimate gene expression rate according to the encounter probability


\section{Discussion}\label{section_discussion}

\section{Figure}\label{section_figures}
%%%%%%%%%%%%%%%%%%%%%%%%%%%%%%%%%%%%%%%%%%%%%%%%%%%%%%%%%%%%%%%%%%%%%%%
\begin{figure}[H]
\begin{subfigure}[b]{0.1\textwidth}
\includegraphics[scale=0.41]{TadDandENoraEtAl2012}
\caption{}
\end{subfigure}

\begin{subfigure}[b]{0.1\textwidth}
\includegraphics[scale=0.16]{meanDataFitTADDAndE}
\caption{}
\end{subfigure}

\begin{subfigure}[b]{0.1\textwidth}
\includegraphics[scale=0.16]{fittedExpValuesWithSplineAverageTADDAndE}
\caption{}
\end{subfigure}

\caption{\textbf{Analysis of the experimental data} (a) Contact histogram from the 5C experiments in 2 discrete regions of high self interactions, termed TAD D and E (Nora et al \cite{Nora2012}) (b) The encounter probability graphs for each of the 307 beads shows long range interactions between beads in TAD E and between TAD D and E, a fit of the form $\alpha Distance ^{-\beta}$ for the mean encounter in each distance (red curve), was found to have $\beta=0.729$, well below the expected $\beta=1.5$ for a linear Rouse polymer, implying compact configuration of the polymer (c) the calculated $\beta$ value for each of the 307 beads}
\label{figure_TADDAndENoraEtAl2012}
\end{figure}
%%%%%%%%%%%%%%%%%%%%%%%%%%%%%%%%%%%%%%%%%%%%%%%%%%%%%%%%%%%%%%%%%%%%%%%


\begin{figure}[H]
\includegraphics[scale=0.2]{meanEncounterMatrixOfSimulatingTADEandDWithLoops}
\caption{\textbf{Simulation of a chain with loops at positions corresponding to the peaks of the encounter data} is insufficient to create an encounter histogram resembling the appearance of two TADs as in the results of the 5C experiment (Figure \ref{figure_TADDAndENoraEtAl2012} a) }
\label{figure_encounterProbabilityPeaksOfTheEncounterData}
\end{figure}

\begin{figure}[H]
\begin{subfigure}[b]{0.1\textwidth}
\includegraphics[scale=0.15]{polymerModelWithOneFixedLoopAndInternalConnections}
\caption{}
\end{subfigure}
~~~~~~~~~~~~~~~~
\begin{subfigure}[b]{0.2\textwidth}
\includegraphics[scale=0.2]{encounterHistogramOneFixedLoop1To10RandomLoops307Beads}
\caption{}
\end{subfigure}

\begin{subfigure}[b]{0.2\textwidth}
\includegraphics[scale=0.2]{fittedExpOneFixedLoop1To10RandomLoops307Beads}
\caption{}
\end{subfigure}

\caption{Encounter profile for a polymer model with one TAD and 0-10 random internal loops. (a) a sketch of the polymer model used in simulations, bead 107 and 207 were connected to form a 100 beads loop, 0 to 10 internal loops were sequentially added in positions chosen by randomly picking two beads in the loop, as an example shown by the orange arrows (b) beads' encounter histograms for the case of 0 to 10 internal loops show the appearance of a TAD in the central region of the polymer corresponding to the position of the fixed big loop (c) the encounter probability of the form $\alpha d^{-\beta}$ was fitted for each of the 11 cases, the $\beta$ values, the mean $\beta$ value shows a linear decrease with the number of random loops (inner axes).}
\label{figure_encounterProfileOneTADWithTails}
\end{figure}

\begin{figure}[H]
\begin{subfigure}[b]{0.01\textwidth}
\includegraphics[scale=0.1]{polymerModelWithTwofixedLoopsAndInternalConnections}
\caption{}
\end{subfigure}
~~~~~~~~~~~~~~~~~~~~~~~~
\begin{subfigure}[b]{0.5\textwidth}
\includegraphics[scale=0.3]{EncounterHistogramsTwoFixedLoops0To10RandomLoops307Beads}
\caption{}
\end{subfigure}

\begin{subfigure}[b]{0.12\textwidth}
\includegraphics[scale=0.2]{fittedExpTwoFixedLoop1To10RandomLoops307Beads}
\caption{}
\end{subfigure}
\caption{\textbf{Encounter profile for a polymer model with two TADs and 1-10 random internal loops in each} (a) a sketch of the polymer model used in simulations, beads 1, 107 and 108, 307 were connected to form two large loops corresponding to the positions of TAD D and E in the experimental data. One to 10 internal loops were sequentially added in each large loops, loop positions were chosen by randomly picking two beads in each large loop, as an example shown by the orange arrows (b) beads' encounter histograms for the case of 0 to 10 internal loops show the appearance of two TADs (c) encounter probability function of the from $\alpha d^{-\beta}$ was fitted for each of the 11 cases, beads in the edge between the two big loops show sharp decrease in $\beta$ duo to the high long range encounter frequency with beads in both TADs. The mean $\beta$ value shows a linear decrease with the number of random loops (inner axes)}
\label{figure_encounterProfileTwoTADs}

\end{figure}


%> the bibliography section
\bibliographystyle{plain}
\bibliography{randomLoopsBibliography} % the bibliography.bib
\end{document}

